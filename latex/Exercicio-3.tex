\section{Exercício 3}


Usando a linguagem Python, criar uma função para calcular $f(x) = \sin(x)$ 
utilizando um polinômio de Taylor em torno de $x_0 = 0$. 
O polinômio deve ter o menor grau possível de modo que o erro da aproximação seja inferior 
a $10^{-7}$ para \textit{qualquer valor} de $x \in \mathbb{R}$. 
Você precisa calcular o número de termos para a sua aproximação utilizando o Teorema de Taylor. 
Teste sua função calculando os valores para $x = \dfrac{9\pi}{4}$, $\dfrac{21\pi}{4}$ e $\dfrac{41\pi}{4}$.\linebreak

\vspace{0.1em}

(\textit{Dica:} note que a função é periódica e portanto o valor de $x$, argumento da sua função, pode ser transformado antes de operar o polinômio. Além disso, você pode calcular valores para comparação usando a função \texttt{sin} da biblioteca \texttt{numpy}.)



