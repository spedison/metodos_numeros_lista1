\documentclass[12pt,oneside]{abntex2}

\usepackage[brazil]{babel}
\usepackage[utf8]{inputenc}
\usepackage{graphicx}
\usepackage{amsmath}
\usepackage{float}
\usepackage{caption}
\usepackage{indentfirst}
\usepackage{setspace}

\setlength{\parindent}{1.25cm}
\setlength{\parskip}{0pt}
\onehalfspacing

\titulo{Relatório Técnico: Análise de Sinais em Java}
\autor{Seu Nome}
\instituicao{Nome da Universidade\\Curso de Engenharia}
\local{Cidade}
\data{\today}

\begin{document}

\imprimircapa
\imprimirfolhaderosto
\tableofcontents
\listoffigures
\cleardoublepage

\begin{resumo}
Este trabalho apresenta um estudo automatizado de análise de sinais. Dados foram gerados via Java, plotados com Gnuplot e organizados em um relatório técnico conforme as normas ABNT com \LaTeX.
\end{resumo}

\section{Introdução}
Este relatório demonstra como combinar ferramentas modernas em um fluxo automatizado com qualidade técnica e científica.

\section{Metodologia}
Utilizamos Java para gerar dados senoidais e Gnuplot para construir o gráfico, integrando ao \LaTeX.

\section{Resultados}
\begin{figure}[H]
\centering
\includegraphics[width=0.75\textwidth]{grafico.jpg}
\caption{Sinal senoidal gerado automaticamente}
\end{figure}

\section{Conclusão}
O processo permite repetir experimentos, gerar novos dados e manter a apresentação profissional.

\section*{Referências}

\noindent
[1] ASSOCIAÇÃO BRASILEIRA DE NORMAS TÉCNICAS. NBR 14724: Trabalhos acadêmicos. Rio de Janeiro, 2011.

\end{document}

