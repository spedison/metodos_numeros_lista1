\section{Conclusão}
Esta lista tem como objetivos revisitar o conteúdo em aula, fazer as implementações e aplicar ajustes em casos pontuais entendendo o funcionamento das funções, rever e enteder as suas peculiaridades e como explorá-las. \\
Alguns ajustes usando características das próprias funções como as simetrias e periodicidade da função $sin(x)$ (no item \ref{sec:caracteristicas-funcao-sin}) ou as limitações de séries que divergem como a função $arctan(x)$ (no item \ref{sec:melhoramentos-serie-arctan}).\\
Um objetivo paralelo e muito interessante foi entender e aplicar o uso destas ferramentas para além de reescrever funções, mas também usá-la para resolver problemas mais complexos como integrais que podem ser simplificadas nesses processo de computação numérica (no item \ref{sec:melhoramentos-serie-arctan}).\\
Como objetivo secundário pode-se contabilizar o aprendizado de algumas tecnologias que facilitam a implementação deste trabalho.
Entre elas : \cite{site-latex}, \cite{site-gradle}, \cite{site-gnuplot-oficial} além de sua documentação extra como \cite{site-gnuplot-documentacao} e a própria linguagem usada para implementação dos exercícios \cite{site-python-org}.