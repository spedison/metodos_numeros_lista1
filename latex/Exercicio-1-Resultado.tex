\subsection{Resultado - Exercício 1}

Os passos são:
\begin{itemize}[leftmargin=3.5em, itemsep=-.5mm, topsep=0.5mm]
    \item Derivar f(x) 4 vezes.
    \item Aplicar a série de Taylor
    \item Gerar dados em Pyhon
    \item Imprimir Gráfico usando GnuPlot
    \item Apresentar o código Python e GnuPlot
 \end{itemize}

\subsubsection{Derivar f(x) 4 vezes}

derivar a função $f(x) = \log(5 + 6x)$ usando a regra da cadeia.
Primeiro, identificamos as funções internas:

\[
    \begin{aligned}
        f(g(x)) &= \log(5 + 6x) \\
        f(x) &= \log(x) \\
        g(x) &= 5 + 6x
    \end{aligned}
\]

Aplicamos a regra da cadeia:

\[
    \frac{d}{dx} f(g(x)) = f'(g(x)) \cdot g'(x)
\]

Calculamos as derivadas individuais:

\[
    \begin{aligned}
        f(x) &= \log(x) \\
        f'(x) &= \frac{1}{x} \\
        g(x) &= 5 + 6\cdot x \\
        g'(x) &= 6
    \end{aligned}
\]

Substituindo na regra da cadeia:

\[
    \frac{d}{dx} f(g(x)) = \frac{1}{5 + 6x} \cdot 6 = \frac{6}{5 + 6x}
\]

Portanto, a derivada da função é:

\[
    f'(x) = \frac{6}{5 + 6x}
\]
Aplicamos a derivada mais uma vez:

\[
    \begin{aligned}
        f''(x) = 6 * \frac{d}{dx} (5 + 6x)^{-1} \\
        f''(x) = - 6 * 6 *  \frac{1}{ (5 + 6x)^{2} } \\
        f''(x) = - \frac{36}{ (5 + 6x)^{2} } \\
    \end{aligned}
\]

Repetindo o processos temos:

\[
    \begin{aligned}
        f'''(x) = \frac{d}{dx} -\frac{36}{(5 + 6x)^{2}} \\
        f'''(x) = - 36 \cdot \frac{d}{dx} (5 + 6x)^{-2} \\
        f'''(x) = 36 \cdot 6 \cdot 2 \cdot (5 + 6x)^{-3} \\
        f'''(x) = 432 \cdot (5 + 6x)^{-3} \\
        f'''(x) = \frac{432}{ (5 + 6x)^{3} } \\
    \end{aligned}
\]

Fazendo a última derivada:

\[
    \begin{aligned}
        f''''(x) =\frac{d}{dx} \frac{432}{ (5 + 6x)^{3} }\\
        f''''(x) = 432 \cdot \frac{d}{dx} \cdot (5 + 6x)^{-3}\\
        f''''(x) = -3 \cdot 6 \cdot 432 \cdot (5 + 6x)^{-4}\\
        f''''(x) = -7776 \cdot (5 + 6x)^{-4}\\
        f''''(x) = \frac{-7776}{(5 + 6x)^{4}}\\
    \end{aligned}
\]

\subsubsection{Aplicar a série de Taylor}

Agora com as derivadas em mãos:

\[
    \begin{aligned}
        f'(x) =    \frac{6}{5 + 6x} \\
        f''(x) =   \frac{-36}{ (5 + 6x)^{2} } \\
        f'''(x) =  \frac{432}{ (5 + 6x)^{3} } \\
        f''''(x) = \frac{-7776}{(5 + 6x)^{4}}\\
    \end{aligned}
\]

Podemos aplicar a Série de Taylor.
\[
    f(x) = \sum_{n=0}^{\infty} \frac{f^{(n)}(x_0)}{n!}(x - x_0)^n
\]

Vamos avaliar o valor em $X_0 = 0$ para cada uma das derivadas.
    \begin{gather*}
        f(0) =    \log(5) = 1.6094 \\
        f'(0) =    \frac{6}{5} \\
        f''(0) =   \frac{-36}{5^{2}}  = \frac{-36}{25} \\
        f'''(0) =  \frac{432}{5^{3}}  = \frac{432}{125} \\
        f''''(0) = \frac{-7776}{5^{4}} = \frac{-7776}{625} \\
    \end{gather*}

Montando as Séries temos:
    \begin{gather*}
        P_0 (x) = f(0) = \log(5) = 1.6094\\
        P_1 (x) = P_0(x) + \frac{\frac{6}{5}}{1!} \cdot x = P_0(x) + \frac{6 \cdot x}{5} \\
        P_2 (x) = P_1(x) + \frac{\frac{-36}{25}}{2!} \cdot x^2 = P_1(x) + \frac{-36 \cdot x^2}{50} \\
        P_3 (x) = P_2(x) + \frac{\frac{432}{25}}{3!} \cdot x^3 = P_2(x) + \frac{432 \cdot x^3}{150} \\
        P_4 (x) = P_3(x) + \frac{\frac{7776}{625}}{4!} \cdot x^3 = P_3(x) + \frac{-7776 \cdot x^4}{15000} \\
    \end{gather*}

\subsubsection{Gerar dados em Pyhon}

Com todos os coeficientes do polinômio podemos codificar e executando o programa em Python com o comando para criar o arquivo com os dados: python exercicio1.py

%%\newpage

\subsubsection{Imprimir Gráfico usando GnuPlot}

Agora com todas as funções definidas, e dados gerados podemos gerar o gráfico para visualizar as diferenças:

\begin{figure}[htbp]
    \centering
    \includegraphics[width=.82\textwidth]{imagens/exercicio1.png}
    \caption{Função $\log(5+6*x)$ aproximada pelo série de taylor em torno de $X_0=0$}
    \label{fig:grafico}
\end{figure}

\newpage

\subsubsection{Apresentar código Python e GnuPlot}

\lstinputlisting[style=python]{scripts/exercicio1.py}
Código para a geração dos dados no arquivo exercício1.csv
\newpage

\lstinputlisting[style=gnuplot]{scripts/exercicio1.gnuplot}
Código para do gráfico do exercício1 baseado nos dados do arquivo exercício1.csv

\newpage





