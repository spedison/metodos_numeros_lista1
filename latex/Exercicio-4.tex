\section{Exercício 4}
Obtenha uma expansão usando polinômios de Taylor para $f(t) = \frac{1}{1 + t^2}$, em torno do ponto $x_0 = 0$. \linebreak

(\textit{Dica:} Use a expansão em polinômio de Taylor para $1/(1 + x)$ e aplique no ponto $x = t^2$). \linebreak
Em seguida, obtenha uma aproximação para $\tan^{-1}(x)$, sabendo que:\linebreak

\[
    \tan^{-1}(x) = \int_0^x \frac{1}{1 + t^2} \, dt.
\]

\vspace{1em}  % espaço opcional para separação

Usando o \textbf{algoritmo de Horner} e um polinômio de grau 10, calcule valores aproximados para $\tan^{-1}(\pi/2)$, $\tan^{-1}(\pi/3)$ e $\tan^{-1}(\pi/4)$ e compare os resultados da sua aproximação com os valores reais (você pode calcular esses valores usando o próprio python com a função \texttt{arctan} da biblioteca \texttt{numpy}).


